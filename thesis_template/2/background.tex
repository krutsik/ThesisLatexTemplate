\chapter{State of the Art}

\section{Industrial Advertising Frameworks}

According to a 2012 article by Matt Marshall, the most reliable companies in mobile advertising, that directly offer a monetization service for the mobile platform, are AdMob, Millennial Media, iAd, Flurry, InMobi, Chartboost, MoPub and Amobee \footnote[2]{Matt Marshall, The top 10 mobile advertising companies, http://venturebeat.com/2013/06/12/the-top-10-mobile-advertising-companies/, 14.05.15}. Following is a short overview of these eight platforms.

AdMob was founded in 2006 and acquired by Google in 2010 \footnote[3]{https://www.crunchbase.com/organization/admob, 14.05.15}. According to research firm eMarketer Google is, by a large margin, the leader of the global digital ad market \footnote[4]{Microsoft to Surpass Yahoo in Global Digital Ad Market Share This Year, http://www.emarketer.com/Article/Microsoft-Surpass-Yahoo-Global-Digital-Ad-Market-Share-This-Year/1011012, 14.05.15}. At the time of Google acquiring AdMob it was one of the market leaders in mobile advertising \footnote[5]{We’ve officially acquired AdMob!, http://googleblog.blogspot.com/2010/05/weve-officially-acquired-admob.html, 14.05.15}. In 2011 it was merged with their existing AdSense platform, which was a solution for web advertising \footnote[6]{Narender Singh, Admob for mobile web merging with Adsense, https://www.techmesto.com/admob-mobile-web-merging-with-adsense/, 14.05.15}. AdSense has software development kits (SDKs) for Android, iOS and Windows Phone \footnote[7]{https://developers.google.com/mobile-ads-sdk/docs/admob/android/banner, 14.05.15}  and supports displaying in-application advertisements as a rectangle, a banner \footnote[8]{https://developers.google.com/mobile-ads-sdk/docs/admob/android/banner, 14.05.15}
or a full screen interstitial \footnote[9]{https://developers.google.com/mobile-ads-sdk/docs/admob/android/interstitial, 14.05.15}.

Millennial Media was founded in 2006 \footnote[10]{https://www.crunchbase.com/organization/millennial-media, 14.05.15}. Apart from Google, it is the only other mobile advertising company in the eMarketer list of top ten global digital advertising companies \footnote[11]{Microsoft to Surpass Yahoo in Global Digital Ad Market Share This Year, http://www.emarketer.com/Article/Microsoft-Surpass-Yahoo-Global-Digital-Ad-Market-Share-This-Year/1011012, 14.05.15}. They offer an end-to-end technology stack with SDKs for Android, iOS, BlackBerry and Symbian \footnote[12]{http://docs.millennialmedia.com/, 14.05.15}. Ads can be displayed as a rectangle, banner or interstitial \footnote[13]{http://docs.millennialmedia.com/android-SDK/, 14.05.15}.

Like Google, in 2010 Apple acquired a mobile advertising company -- Quattro, founded in 2006 \footnote[14]{https://www.crunchbase.com/organization/quattro-wireless, 14.05.15}. It was then rebranded as iAd \footnote[15]{Eric Jackson, What Is Next For Apple's iAd?, http://www.forbes.com/sites/ericjackson/2011/11/11/what-is-next-for-apples-iad/, 14.05.15}, which has not seen much success to date and, according to an article posted on Forbes, only accounts for 2.5\% of mobile advertising revenues in the United States \footnote[16]{How Apple Is Revamping Its iAd Platform, http://www.forbes.com/sites/greatspeculations/2014/12/18/how-apple-is-revamping-its-iad-platform/, 14.05.15}. iAd only supports iOS and the advertisements are native, meaning that the developer can define the bounds of the advertisement with x and y coordinates, width and height \footnote[17]{Chris Ching, iAd Tutorial – How To Integrate iAd Banners Into Your App, http://codewithchris.com/iad-tutorial/, 14.05.15}.

Flurry was founded in 2005 and acquired by Yahoo in July of 2014 \footnote[18]{https://www.crunchbase.com/organization/flurry, 14.05.15}. The main selling point of Flurry is its analytics tool with the largest data-set on application usage \footnote[19]{Raúl Castañón, Yahoo Agrees To Buy Mobile Ad Analytics Firm Flurry, http://maps.yankeegroup.com/ygapp/content/f47515e9c15c4eac90bab1ced3582bdf/50/DAILYINSIGHT/0, 14.05.15}. Flurry supports Android and iOS and the advertisements can be displayed as either a banner or full screen.

InMobi was founded in 2007 \footnote[20]{https://www.crunchbase.com/organization/inmobi, 14.05.15}. They claim to be the first advertising platform with 1 billion unique users \footnote[21]{http://www.inmobi.com/, 14.05.15}. InMobi supports Android, iOS and Windows Phone \footnote[22]{http://www.inmobi.com/monetize/, 14.05.15}

\newpage

and the advertisements can be displayed as a banner, full screen or native \footnote[23]{https://www.inmobi.com/support/integration/23817448/22051163/android-sdk-integration-guide/, 14.05.15}.

Chartboost was founded in 2011 \footnote[24]{https://www.crunchbase.com/organization/chartboost, 14.05.15}. It is a mobile games-only advertisement network and focuses on cross promotion of mobile games, meaning that most of the advertisements are about downloading other games. According to their own words it only takes ten lines of code to integrate Chartboost into an application \footnote[25]{https://www.chartboost.com/, 14.05.15}. It supports Android and iOS and the advertisements can only be displayed full screen \footnote[26]{Alvaris Falcon, 20 Advertising Networks to Monetize Your Mobile App, http://www.hongkiat.com/blog/mobile-app-monetizing-networks/, 14.05.15}.

MoPub was founded in 2010 and acquired by Twitter in 2013 \footnote[27]{https://www.crunchbase.com/organization/mopub, 14.05.15}. Their main selling point is a large real-time bidding exchange for in-application advertisements. This means that there is no sales person between the advertiser and the developer. Advertisers bid directly for the opportunity to have their advertisements displayed in an application and developers have  control and transparency over the ads that are delivered into the application \footnote[28]{http://www.mopub.com/platform/marketplace/, 14.05.15}. Supported platforms are Android and iOS. Supported formats are banner, full screen and native \footnote[29]{http://www.mopub.com/resources/, 14.05.15}.

Amobee was founded in 2005 and acquired by SingTel in 2012 \footnote[30]{https://www.crunchbase.com/organization/amobee, 14.05.15}. They have SDKs for Android, iOS, Windows Phone and BlackBerry. Possible formats are banners and full screen video.

Common for all these advertising platforms is the fact that none of them offer a format where the user can interact with the advertisement apart from tapping on it and, in the case of full screen advertisements, close it. That is not to say that attempts have not been made to make advertisements interactable and more fun for the user. Such frameworks will be described in the following section.

\section{Related Works}

Some advertisement providers have taken to using push notifications to deliver their advertisements \footnote[31]{http://www.airpush.com/, 14.05.15}, however the delivery of the advertisements is static as with all conventional frameworks. Moreover, they are not concerned with the quality of the advertisements as much as the quantity \footnote[32]{Quentyn Kennemer, Spammy ads in the notification bar die this week as Google’s latest Play Store changes take effect, http://phandroid.com/2013/09/30/google-play-notification-ads-policy/, 14.05.15}.

In 2009 apparel maker Dockers San Francisco launched an advertisement for iPhone, which used the phone's accelerometer to respond to the user shaking their phone, which would then make a dancer, appearing on the screen, perform his moves. Dockers themselves believe that it was the first motion sensitive mobile advertisement. It was featured in iPhone games “iBasketball”, “iGolf” and “iBowl” \footnote[33]{Rita Chang, Dockers Introduces 'Shakeable' iPhone Ad, http://adage.com/article/digital/iphone-marketing-dockers-introduces-shakeable-ad/135197/, 14.05.15}.

Since then many advertisers have taken to offering advertisements using a phone’s sensors to make them more interesting to the user. Most notable perhaps is Adtile, which uses advertisements built with standard web technologies, using a JavaScript Motion Framework which gives access, among other things, to the device’s sensors and GPS, to make their advertisements fully interactable. Adtile supports Android and iOS devices \footnote[34]{http://www.adtile.me/motion-ads/, 14.05.15}.

Multiple research works have proposed different strategies to overcome the issues that mobile advertisements and advertising mechanisms have.

Several articles discuss mechanisms for providing advertisements to the user based on their location, which would make the content of advertisements more relevant due to the proximity of advertised locations to the user \cite{hristova2004adme}\cite{aalto2004btandwap}. However, the user might not have location tracking enabled on their mobile device, which reduces the usefulness of such mechanisms.

In other fields a lot of research has been done on methods of sensing the user's mood and emotions\cite{geller2014howdoyoufeel}\cite{kiel2004frustration}. There are also articles describing using such methods and sending advertisements based on that information \cite{hristova2004adme}\cite{likamwa2013moodscope}. However, the recognition of the user's emotions requires constant and intensive processing and computation. This quickly drains the mobile device's battery, hindering the practice quite ineffective.

A company called In-mobile proposes a framework that would split advertisements into smaller pieces and embed them into 2D objects, such as obstacles or collectable items, in a level of a mobile game. The advertisement would then be displayed to the user over a period of time, without taking away any extra screen real estate.

Based on experimental results, their proposed mechanism improved the users' perception of mobile advertisements by 60\% and increased the effectiveness of an advertisement by 30\%. Unfortunately, this method of displaying advertisements is only applicable for mobile games and no other kind of applications \footnote[35]{http://in-mobilelabs.com/, 14.05.15}.

\section{Summary}

The most commonly used mechanisms for displaying advertisements do so statically and are disruptive to the experience of using the application. Attempts have been made to counter the problems of these mechanisms, but such solutions are often difficult to integrate into an application or have other drawbacks. The next section describes the problem in more detail.