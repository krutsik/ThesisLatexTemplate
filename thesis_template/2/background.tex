% this file is called up by thesis.tex
% content in this file will be fed into the main document

\chapter{Background} % top level followed by section, subsection

%transition between chapters, usually no more than two paragraphs

% ----------------------- State of the art ------------------------

\section{Classical Advertising Frameworks}

According to a 2012 article by Matt Marshall, the most reliable companies in mobile advertising, that directly offer a monetization service for the mobile platform, are AdMob, Millennial Media, iAd, Flurry, InMobi, Chartboost, MoPub and Amobee. \cite{mmarshall-top10} Following is a short overview of these eight platforms.

AdMob was founded in 2006 and acquired by Google in 2010. \cite{crunchbase-admob} According to research firm eMarketer Google is, by a large margin, the leader of the global digital ad market. \cite{emarketer-google} At the time of Google acquiring AdMob it was one of the market leaders in mobile advertising. \cite{google-admob} In 2011 it was merged with their existing AdSense platform, which is a solution for web advertising. \cite{admob-adsense} AdSense has software development kits (SDKs) for Android, iOS and Windows Phone \cite{admob-sdk}  and supports displaying in-app advertisements as a rectangle, a banner \cite{admob-banner} or a full screen interstitial.\cite{admob-interstitial}

Millennial Media was founded in 2006. \cite{crunchbase-millenial} Apart from Google, it is the only other mobile advertising company in the eMarketer list of top ten global digital advertising companies. \cite{emarketer-google} They offer an end-to-end technology stack with SDKs for Android, iOS, BlackBerry and Symbian. \cite{millenial-sdk} Ads can be displayed as a rectangle, banner or interstitial. \cite{millenial-android}

Like Google, in 2010 Apple acquired a mobile advertising company -- Quattro, founded in 2006. \cite{crunchbase-quattro} It was then rebranded as iAd \cite{forbes-iad}, which has not seen much success to date and, according to an article posted on Forbes, only accounts for 2.5\% of mobile advertising revenues in the United States. \cite{forbes-iad2} iAd only supports iOS and the advertisements are native, meaning that the developer can define the bounds of the advertisement with x and y coordinates, width and height. \cite{iad-tutorial}

Flurry was founded in 2005 and acquired by Yahoo in July of 2014. \cite{crunchbase-flurry} The main selling point of Flurry is its analytics tool with the largest data-set on application usage. \cite{maps-yankeegroup} Flurry supports Android and iOS and the advertisements can be displayed as either a banner or full screen.

InMobi was founded in 2007. \cite{crunchbase-inmobi} They claim to be the first advertising platform with 1 billion unique users. \cite{inmobi} InMobi supports Android, iOS and Windows Phone \cite{inmobi-monetize} and the advertisements can be displayed as a banner, full screen or native. \cite{inmobi-sdk}

Chartboost was founded in 2011. \cite{crunchbase-chartboost} It is a mobile games-only advertisement network and focuses on cross promotion of mobile games, meaning that most of the advertisements are about downloading other games. According to their own words it only takes ten lines of code to integrate Chartboost into an application. \cite{chartboost} It supports Android and iOS and the advertisements can only be displayed full screen.\cite{hongikiat-monetize}

MoPub was founded in 2010 and acquired by Twitter in 2013. \cite{crunchbase-mopub} Their main selling point is a large real-time bidding exchange for in-application advertisements. This means that there is no sales person between the advertiser and the developer. Advertisers bid directly for the opportunity to have their advertisements displayed in an application and developers have  control and transparency over the ads that are delivered into the application. \cite{mopub-marketplace} Supported platforms are Android and iOS. \cite{mopub-resources} Supported formats are banner, full screen and native.

Amobee was founded in 2005 and acquired by SingTel in 2012. \cite{crunchbase-amobee} They have SDKs for Android, iOS, Windows Phone and BlackBerry. Possible formats are banners and full screen video.

Common for all these advertising platforms is the fact that none of them offer a format where the user can interact with the advertisement apart from tapping on it and, in case of full screen ads, close it. That is not to say that attempts haven’t been made to make advertisements interactable and/or more fun for the user.

\section{Dynamic Advertising Frameworks}

In 2009 apparel maker Dockers San Francisco launched an advertisement for iPhone, which used the phone’s accelerometer to respond to the user shaking their phone, which would then make a dancer, appearing on the screen, perform his moves. Dockers themselves believe that it was the first motion sensitive mobile advertisement. It was featured in iPhone games “iBasketball”, “iGolf” and “iBowl”.\cite{dockers}

Since then many advertisers have taken to offering advertisements using the phone’s sensors to make them more interesting to the user. Most notable perhaps is Adtile, which uses advertisements built with standard web technologies, using a JavaScript Motion Framework which gives access, among other things, to the device’s sensors and GPS, to make their advertisements fully interactable. Adtile supports Android and iOS devices. \cite{adtile}

But sometimes, even though the advertisements are interesting, the user just wants to continue using the application they were using with minimal interruption by displayed advertisements. An article titled “Can In-App Ads be Less Annoying for Smartphone Users?” by Huber Flores, Pan Hui and Yong Li proposes a framework that would split advertisements into smaller pieces and embed them into 2D objects, such as obstacles or collectable items, in a level of a mobile game. The advertisement would then be displayed to the user over a period of time, without taking away any extra screen real estate. \cite{gesture-ads}

\section{Related Works}

\section{Summary}

As it stands, there are no well known, easy to implement libraries to display responsive advertisements in classical mobile applications. Next section describes the proposed solution in detail.

% ---------------------------------------------------------------------------
% -----------------------end of thesis sub-document----------------
% --------------------------------------------------------------------------- 