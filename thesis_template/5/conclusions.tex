\chapter{Conclusions}

Sending personalised advertisements to a user's smartphone is one the most intimate advertising channels. The excessive use of it, however, has become a serious problem. Excessive amount of Advertisements in mobile applications degrade the user experience and make the user view the advertisements in a negative light. The effectiveness of such advertising questionable as well.

A library was developed that presents advertisements in mobile applications in a more user friendly manner. It lets the user move it around on the screen according to need or move it off the screen altogether. If the user chooses to keep it on the screen for later viewing, the advertisement is small enough that it does not distract from using the application. In addition, users themselves can give feedback on the advertisements, which more accurately helps send the user more relevant advertisements in the future, instead of relying on profiling algorithms to decide the user's likes and dislikes.

However, the developed solution is just a proof of concept and has far a way to go for being usable in an actual real world solution. The client side of the library needs stability improvements and rigorous testing. The animations could be improved and the user interface made more intuitive to use.

The server-side needs to be built from scratch, with improvements in security and stability. Also, keeping in mind that an advertising framework needs to handle a large amount of users at any given time, which the current solution can not.

All in all, the feedback received from the control group was mainly positive. This gives hope that, in the future, a state can be reached, where mobile advertisements are construed as a useful and positive thing by an application user, rather than an annoyance and encourages exploring new directions in mobile advertising.