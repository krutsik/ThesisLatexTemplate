\begin{abstracts}

Tänapäeval on reklaamid on lahutamatu osa mobiilirakenduste arendusest. Paljudel juhtudel lasevad reklaamid arendajal oma rakendust kasutajatele tasuta pakkuda. Siiski, reklaamid mobiilirakendustes segavad sageli kasutajat ning sageli muudavad rakenduse kasutamise nii ebamugavaks, et kasutaja otsustab rakenduse kasutamise sootuks lõpetada.

Väljapakutud lahendus sellele probleemile on näidata rekandusesisest reklaami kasutajale dünaamilisemalt kujul. Sellisel moel, et reklaam tunduks nagu osa rakendusest, mitte ebameeldiv lisa sellele.

Hüppoteesi kontrollimiseks arendatakse selline mehanism Androidi teegi, ning kaasaskäiva serverirakenduse, kujul ning lisatakse see ühele mobiilirakendusele.

Tulemuste põhjal arvas 64\% kasutajatest, kes rakendust proovis, et see mehanism on vähem häiriv kui klassikalised meetodid reklaami näitamiseks. 52\% kasutajatest väitis end olevat reklaami sisust rohkem huvitatud.

Need tulemust näitavad, kui ebaeffektiivsed on klassikalised reklaamid mobiilirakendustes ning julgustavas uurima uusi suundi mobiilse reklaami näitamiseks.

\bigskip

\textbf{Võtmesõnad:} Mobiilne, Android, Reklaam, Käeliigutused

\end{abstracts}