\begin{abstracts}

\textbf{D\"{u}naamilised Reklaamid Mobiilirakendustele}

T\"{a}nap\"{a}eval on reklaamid lahutamatu osa mobiilirakenduste arendusest. Paljudel juhtudel lasevad reklaamid arendajal oma rakendust kasutajatele tasuta pakkuda. Siiski, reklaamid mobiilirakendustes segavad sageli kasutajat ning sageli muudavad rakenduse kasutamise nii ebamugavaks, et kasutaja otsustab rakenduse kasutamise sootuks l\~{o}petada.

V\"{a}ljapakutud lahendus sellele probleemile on n\"{a}idata rekandusesisest reklaami kasutajale d\"{u}naamilisemal kujul. Sellisel moel, et reklaam tunduks nagu osa rakendusest, mitte selle ebameeldiv lisa.

H\"{u}poteesi kontrollimiseks arendatakse selline mehhanism Androidi teegi, ning kaasask\"{a}iva serverirakenduse, kujul ning lisatakse see \"{u}hele mobiilirakendusele.

Tulemuste p\~{o}hjal arvas 64\% kasutajatest, kes rakendust proovis, et see mehhanism on v\"{a}hem h\"{a}iriv kui klassikalised meetodid reklaami n\"{a}itamiseks. 52\% kasutajatest v\"{a}itis end olevat reklaami sisust rohkem huvitatud.

Need tulemust n\"{a}itavad, kui ebaefektiivsed on klassikalised reklaamid mobiilirakendustes ning julgustavad uurima uusi suundi mobiilse reklaami n\"{a}itamiseks.

\bigskip

\textbf{V\~{o}tmes\~{o}nad:} Mobiilne, Android, Reklaam, K\"{a}eliigutused

\end{abstracts}