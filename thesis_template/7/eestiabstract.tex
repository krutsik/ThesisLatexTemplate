\begin{abstracts}

\textbf{D\"{u}naamilised Reklaamid Mobiilirakendustele}

T\"{a}nap\"{a}eval on reklaamid lahutamatu osa mobiilirakenduste arendusest. Paljudel juhtudel lasevad reklaamid arendajal oma rakendust kasutajatele tasuta pakkuda. Siiski, reklaamid mobiilirakendustes segavad sageli kasutajat ning sageli muudavad rakenduse kasutamise nii ebamugavaks, et kasutaja otsustab rakenduse kasutamise sootuks l\~{o}petada. Lisaks on baseerub reklaamide sisu ebausaldusv\"{a}\"{a}rsete algoritmide tulemustel ning ei pruugi kajastada kasutaja tegelikke huve.

V\"{a}ljapakutud lahendus sellele probleemile on n\"{a}idata rekandusesisest reklaami kasutajale d\"{u}naamilisemal kujul; sellisel moel, et reklaam tunduks nagu osa rakendusest, mitte selle ebameeldiv lisa; ning lasta kasutajatel endil enda tagasisidet selle kohta, millised reklaamid neili huvi pakuvad ning millised mitte.

H\"{u}poteesi kontrollimiseks arendatakse selline mehhanism Androidi teegi, ning kaasask\"{a}iva serverirakenduse kujul. Kasutusloona lisatakse see \"{u}hele mobiilirakendusele, et demonstreerida sellise lähenemise kasulikkust.

Tulemuste p\~{o}hjal arvas 64\% kasutajatest, kes rakendust proovis, et see mehhanism on v\"{a}hem h\"{a}iriv kui klassikalised meetodid reklaami n\"{a}itamiseks. 52\% kasutajatest v\"{a}itis end olevat reklaami sisust rohkem huvitatud.

Need tulemust n\"{a}itavad, kui ebaefektiivsed on klassikalised reklaamid mobiilirakendustes ning julgustavad uurima uusi suundi mobiilse reklaami n\"{a}itamiseks.

\bigskip

\textbf{V\~{o}tmes\~{o}nad:} Mobiilne, Android, Reklaam, K\"{a}eliigutused

\end{abstracts}