\begin{abstracts}

\textbf{Dünaamilised Reklaamid Mobiilirakendustele}

T\"{a}nap\"{a}eval on reklaamid on lahutamatu osa mobiilirakenduste arendusest. Paljudel juhtudel lasevad reklaamid arendajal oma rakendust kasutajatele tasuta pakkuda. Siiski, reklaamid mobiilirakendustes segavad sageli kasutajat ning sageli muudavad rakenduse kasutamise nii ebamugavaks, et kasutaja otsustab rakenduse kasutamise sootuks l\~{o}petada.

V\"{a}ljapakutud lahendus sellele probleemile on n\"{a}idata rekandusesisest reklaami kasutajale d\"{u}naamilisemalt kujul. Sellisel moel, et reklaam tunduks nagu osa rakendusest, mitte ebameeldiv lisa sellele.

H\"{u}ppoteesi kontrollimiseks arendatakse selline mehanism Androidi teegi, ning kaasask\"{a}iva serverirakenduse, kujul ning lisatakse see \"{u}hele mobiilirakendusele.

Tulemuste p\~{o}hjal arvas 64\% kasutajatest, kes rakendust proovis, et see mehanism on v\"{a}hem h\"{a}iriv kui klassikalised meetodid reklaami n\"{a}itamiseks. 52\% kasutajatest v\"{a}itis end olevat reklaami sisust rohkem huvitatud.

Need tulemust n\"{a}itavad, kui ebaeffektiivsed on klassikalised reklaamid mobiilirakendustes ning julgustavas uurima uusi suundi mobiilse reklaami n\"{a}itamiseks.

\bigskip

\textbf{V\~{o}tmes\~{o}nad:} Mobiilne, Android, Reklaam, K\"{a}eliigutused

\end{abstracts}