
% this file is called up by thesis.tex
% content in this file will be fed into the main document

%: ----------------------- introduction file header -----------------------
\chapter{Introduction}

% the code below specifies where the figures are stored
\ifpdf
    \graphicspath{{1_introduction/figures/PNG/}{1_introduction/figures/PDF/}{1_introduction/figures/}}
\else
    \graphicspath{{1_introduction/figures/EPS/}{1_introduction/figures/}}
\fi

% ----------------------------------------------------------------------
%: ----------------------- introduction content -----------------------
% ----------------------------------------------------------------------



%: ----------------------- HELP: latex document organisation
% the commands below help you to subdivide and organise your thesis
%    \chapter{}       = level 1, top level
%    \section{}       = level 2
%    \subsection{}    = level 3
%    \subsubsection{} = level 4
% note that everything after the percentage sign is hidden from output

Mobile advertising is an inseparable part in today's mobile application development. In many cases advertising enables the developer to offer their application to users for free. The most common way advertisements are displayed to the user is taking up a big part, if not all, of the device's screen real estate giving the developer little control of the content of the advertisements; often tricking users into clicking the link within the advertisement instead of closing it.

\section{Motivation}
More often than not, in-application advertisements prove to be distracting to the user and, in extreme cases, even degrade the user experience to a point where the user decides to stop using the application.  In addition, usually the user tends to just ignore the advertisements. This makes the advertising model ineffective and compromises the reputation of the application, as well as the developer, without significant gain for any party.

Advertisements influence people's attitude towards advertising by a great deal. When advertisements employ techniques that irritate or bother the user, then the user is likely to perceive them as an unwanted influence.\cite{chowdhury2010consumer} A less intrusive and more intuitive, user feedback based method of displaying advertisements would improve the user experience of the application with the added benefit of users being more likely to see the content of the advertisement.

\section{Contributions}
The goal of this project is to develop a Java library for Android \cite{android:platform} platform that application developers can use, which would integrate advertisements more seamlessly into the flow of the application. It would allow the application to receive all the necessary data via a push notification from the server and then display a very minimalist form of it to the user, without taking up too much of the very limited screen space of the device. The application should still retain usability for any time-critical activities with the advertisement being movable if necessary.

The user can focus their attention on the advertisement when the time is most suitable. They can then use a spread gesture to view a more detailed description of the advertisement or flick it either left to show disinterest or right to show interest. Data will then be sent back to the server to be processed and stored. Showing interest in the advertisement will also store it for future viewing, should the user wish to do so. If the user chooses to ignore the advertisement, it will remain on the screen for a period of time and then disappear.

The server application can then take into account, among other things, the user's preferred topics and location to send advertisements more relevant to the user.

By the end, the library will serve three purposes:
\begin{itemize}
  \item It will make the application more enjoyable for the user, by offering less intrusive methods of advertising and advertisements more relevant to the individual user.
  \item It will make incorporating mobile advertisements easier for the developer and give the developer more freedom to focus on user experience, without having to worry about advertising mechanics.
  \item It will allow the advertiser to offer content relevant to the user, thereby increasing the user's perception of the product as well as the likelihood of the user choosing to learn more about the offer.
\end{itemize}

\section{Outline}
\noindent \textbf{Chapter 2}: describes the most popular mobile advertising frameworks used today and discusses the pros and cons of each. It also takes a closer look at some of the more unorthodox advertising frameworks that do not follow the most common patterns of mobile advertising and take user experience regarding mobile advertisements to the next level.

\noindent \textbf{Chapter 3}: describes in detail the development process and the final structure of the font-end library as well as the server application used to test the library.

\noindent \textbf{Chapter 4}: takes a closer look at user feedback regarding the developed library as well as mobile advertisements in general. Brief analysis.