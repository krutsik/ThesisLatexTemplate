% this file is called up by thesis.tex
% content in this file will be fed into the main document

%: ----------------------- name of chapter  -------------------------
\chapter{Case Studies} % top level followed by section, subsection


%transition between chapters, usually no more than two parragraphs
An application's user is generally not happy about seeing advertisements in the application. They degrade the overall experience and distract from the contents of the application itself. However, for some developers, advertisements are the only way to monetize their application.

In this chapter we try to determine how the proposed advertising methods compares to the more traditional ones in terms of user experience.

%: ----------------------- paths to graphics ------------------------

% change according to folder and file names
\ifpdf
    \graphicspath{{X/figures/PNG/}{X/figures/PDF/}{X/figures/}}
\else
    \graphicspath{{X/figures/EPS/}{X/figures/}}
\fi

%: ----------------------- contents from here ------------------------

%\input{4/File1}	
%\input{4/File}

\section{Setup and Methodology}

To test the hypothesis, the library was implemented to an open-source Android application for reading Wikipedia articles. It is a good example of the possible use cases of the library, because most of the screen space is covered with content and more traditional advertising methods with a static location can bother and distract the user.

To measure how users view the proposed method, a questionnaire was composed. [Appendix A] It consists of 14 questions. 10 of the questions are about day-to-day application usage and opinion on traditional mobile advertisements. The final four questions are pertaining to the dynamic nature of the advertisement and how it compares to advertisements the participants are used to seeing in applications.

There were 25 participants between the ages of 20 and 35, all day-to-day smartphone users. 64% of the participants were male.

They were asked to answer the first ten questions. Then they were asked to read an article of their choice from the Wikipedia application, half way though which an advertisement appeared, and asked to answer the final four questions.

\section{Participants' Opinion on Traditional Mobile Advertising}

The participants claimed to be using their mobile devices anywhere between less than 30 minutes and more than 3 hours, however more than half of the participants use their mobile device 30 minutes to 1 hour each day. [Appendix B]

The average rating the participants gave to the amount of mobile advertisements bothering them is 3.72 on a scale from 1 to 5, 1 meaning not at all and 5 meaning very much. None of the participants claimed that mobile advertisements did not bother them at all. [Appendix C] Many reasons were pointed out why advertisements are bothersome. The main ones being that the advertisements are distracting from using the application (flashing, sounds), take up a lot of screen space and are sometimes difficult or even impossible to get rid of.

Furthermore, 84\% of the participants claim to have uninstalled a mobile application just because of the intrusiveness of it's edvertisements [Appendix D], while only 28\% claim to have looked up a product or service because of an advertisement in an application [Appendix E] and only 8\% to have actually payed for a product or service they saw in an advertisement. [Appendix F]

This shows how ineffective traditional advertising methods are and that people are quite easily willing to stop using an application that has intrusive adverts. Traditional advertising methods are often quantity-over-quality and developers can easily damage their reputation without significant gain.

\section{Participants' Willingness to Adapt to New Methods of Advertising}

The participants were also asked questions about their willingness to use functionality that this library provides without having used it before, to see whether they would be willing to adapt to dynamic advertisements in the applications they use.

All the participants thought they would get rid of an advertisement as soon as it appeared at least half the time [Appendix G], but 44\% of participants thought they would move an advertisement for later viewing at least some of the times and 24\% that they would want to view an advertisement later about half the times. [Appendix H]

When asked how they themselves think advertisements should be displayed in applications to make them less disruptive, the most common answers were that the advertisements should be smaller, blend in with the surroundings more and not be on top of content.

As seen from this feedback, the users would, at least some of the time, be willing to use the functionality this library provides for advertisements. Also, it has many of the features the users themselves proposed for making adverts more user friendly, like being smaller and not being on top of content the users wishes to interact with.

\section{Participants' Opinion of Dynamic Advertisements}

In terms of intrusiveness, all participants thought that the proposed method is less or just as intrusive as the classical method of advertisement delivery. 24\% thought that it is considerably less intrusive and 36\% that is just as intrusive. [Appendix J]

48\% of participants were just as interested in the advertised product as they would have been with the classical advertising method. 40\% were a little bit more interested and the product. 12\% were considerably more interested.[Appendix K]

The general rating participants gave to the method of advertisement delivery is 3.96 on a scale from 1 to 5. [Appendix L]

\section{Summary}

The feedback was mainly positive. Almost all users found that being able to move an advertisement around on the screen is something that they would like to do in mobile applications. Some found that, being used to classical advertisements, the fact that the advertisement can be interacted with is not very intuitive and needs instructions on first application launch.

Some of the participants are application developers and were more than happy to start using this method for application monetization if it ever became a feasible option.

% ---------------------------------------------------------------------------
%: ----------------------- end of thesis sub-document ------------------------
% ---------------------------------------------------------------------------