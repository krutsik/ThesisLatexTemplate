\begin{abstracts}

\textbf{Gesture Ads for Mobile Applications}

Mobile advertising is an inseparable part in today's mobile application development. However, more often than not, in-application advertisements prove to be distracting to the user and, in extreme cases, even degrade the user experience to a point where the user decides to stop using the application. In addition, the content of the advertisements is based on unreliable profiling algorithms and might not reflect the user's actual interests.

A proposed solution to this problem is to display in-application advertisements to the user in a more dynamic way, in a way that would feel like the advertisement is part of the application rather than an unwelcome addition, and allowing users themselves to provide feedback on the topics of advertisements they like and dislike.

To validate the hypothesis, a mechanism is developed in the form of an Android library and an accompanying server application. As a use case it is integrated into an Android application to demonstrate the feasibility of the proposed approach.

According to the results, 64\% of the users that tried out the application found this mechanism to be less intrusive than conventional advertising methods. 52\% of the users reported being more interested in the advertised product.

These results show the ineffectiveness of conventional mobile advertisements and encourage exploring new directions for displaying advertisements in mobile applications.

\bigskip

\textbf{Keywords:} Mobile, Android, Advertising, Gestures

\end{abstracts}