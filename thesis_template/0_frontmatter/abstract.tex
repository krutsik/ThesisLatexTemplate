
% Thesis Abstract -----------------------------------------------------

\begin{abstracts}        %this creates the heading for the abstract page

Mobile advertising is an inseparable part in today's mobile application development. In many cases advertising enables the developer to offer their application to users for free. However, more often than not, in-application advertisements prove to be distracting to the user and, in extreme cases, even degrade the user experience to a point where the user decides to stop using the application. In any case, the user most often tends to just ignore the advertisements, which makes the advertising model ineffective and compromises the reputation of the application without significant gain for any party.

The goal of this project is to develop a library that application developers can use, which would integrate advertisements more seamlessly into the flow of the application. It would allow the application to receive all the necessary data via a push notification from the server and then display a very minimalist form of it to the user, without taking up too much of the very limited screen space of the device. The application will then hopefully still retain usability for any time-critical activities.

The user can focus his/her attention on the advertisement when the time is most suitable (e.g. they have finished reading a paragraph). The user can then use a spread motion to view more a detailed description of the advertisement or flick it either left to show disinterest or right to show interest – a motion many mobile users are already conceptually familiar with. Data will then be sent back to the server to be processed and, based on the processed data, the future advertisements can be more suited to the user's interests. Showing interest in the advertisement will also store it for future viewing, should the user wish to do so. If the user chooses to ignore the advertisement, it will remain on the screen for a period of time and then disappear.

Another use case would be to incorporate the user's current location to the process. This would allow the user to receive advertisements not only relevant in terms of their interests, but also in terms of their current location.

By the end, the library will hopefully serve three purposes:

\begin{itemize}
  \item It will make the application more enjoyable for the user, by offering less intrusive methods of advertising and advertisements more relevant to the individual user.
  \item It will make incorporating mobile advertisements easier for the developer and give the developer more freedom to focus on user experience, without having to worry about advertising mechanics.
  \item It will allow the advertiser to offer content relevant to the user, thereby increasing the user's perception of the product as well as the likelihood of the user choosing to learn more about the offer.
\end{itemize}

\end{abstracts}

% ---------------------------------------------------------------------- 