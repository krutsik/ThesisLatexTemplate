\begin{abstracts}

\textbf{Gesture Ads for Mobile Applications}

Mobile advertising is an inseparable part in today's mobile application development. In many cases advertising enables the developer to offer their application to users for free. However, more often than not, in-application advertisements prove to be distracting to the user and, in extreme cases, even degrade the user experience to a point where the user decides to stop using the application. In addition, the content of the advertisements is based on unreliable profiling algorithms.

A proposed solution to this problem is to display in-application advertisements to the user in a more dynamic way, in a way that would feel like the advertisement is part of the application rather than an unwelcome addition, and allowing users themselves to provide feedback on the topics of advertisements they like and dislike.

To validate the hypothesis

To test the hypothesis, such a mechanism, in the form of an Android library and an accompanying server application, is developed and integrated into a mobile application.

According to the results, 64\% of the users that tried out the application found this mechanism to be less intrusive than classical advertising methods. 52\% of the users reported being more interested in the advertised product.

These results show the ineffectiveness of classical mobile advertisements and encourage exploring new directions for displaying advertisements in mobile applications.

\bigskip

\textbf{Keywords:} Mobile, Android, Advertising, Gestures

\end{abstracts}