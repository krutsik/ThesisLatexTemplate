\noindent\textbf{\large Gesture Ads for Mobile Applications}
\vspace*{2ex}
{\flushleft{\textbf{Abstract:}}}

\noindent Mobile advertising is an inseparable part in today's mobile application development. However, often in-application advertisements prove to be distracting to the user. In addition, the content of the advertisements is based on unreliable profiling algorithms and might not reflect the user's actual interests.
A proposed solution to this problem is to display in-application advertisements to the user in a more dynamic way and allowing users themselves to provide feedback on the topics of advertisements they like and dislike.
To validate the hypothesis, a mechanism is developed in the form of an Android library and an accompanying server application. As a use case it is integrated into an Android application to demonstrate the feasibility of the proposed approach.
According to the results, 64\% of the users that tried out the application found this mechanism to be less intrusive than conventional advertising methods. 52\% of the users reported being more interested in the advertised product.
These results show the ineffectiveness of conventional mobile advertisements and encourage exploring new directions for displaying advertisements in mobile applications.

\vspace*{2ex}

{\flushleft{\textbf{Keywords:} Mobile, Android, Advertising, Gestures}}

\vspace*{3ex}

\noindent\textbf{\large D\"{u}naamilised Reklaamid Mobiilirakendustele}
\vspace*{2ex}
{\flushleft{\textbf{L\"{u}hikokkuv\~{o}te:}}}

\noindent T\"{a}nap\"{a}eval on reklaamid lahutamatu osa mobiilirakenduste arendusest. Paljudel juhtudel lasevad reklaamid arendajal oma rakendust kasutajatele tasuta pakkuda. Siiski, reklaamid mobiilirakendustes segavad sageli kasutajat. Lisaks baseerub reklaamide sisu ebausaldusv\"{a}\"{a}rsete algoritmide tulemustel ning ei pruugi kajastada kasutaja tegelikke huve.
V\"{a}ljapakutud lahendus sellele probleemile on n\"{a}idata rekandusesisest reklaami kasutajale d\"{u}naamilisemal kujul ning lasta kasutajatel endil anda tagasisidet selle kohta, millised reklaamid neile huvi pakuvad ning millised mitte.
H\"{u}poteesi kontrollimiseks arendatakse selline mehhanism Androidi teegi, ning kaasask\"{a}iva serverirakenduse, kujul. Kasutusloona lisatakse see \"{u}hele mobiilirakendusele, et demonstreerida sellise l\"{a}henemise kasulikkust.
64\% rakendust proovinud kasutajatest arvas, et see mehhanism on v\"{a}hem h\"{a}iriv kui klassikalised meetodid reklaami n\"{a}itamiseks. 52\% kasutajatest v\"{a}itis end olevat reklaami sisust rohkem huvitatud.
Need tulemust n\"{a}itavad, kui ebaefektiivsed on klassikalised reklaamid mobiilirakendustes ning julgustavad uurima uusi suundi mobiilse reklaami n\"{a}itamiseks.

\vspace*{2ex}

{\flushleft{\textbf{V\~{o}tmes\~{o}nad:} Mobiilne, Android, Reklaam, K\"{a}eliigutused}}